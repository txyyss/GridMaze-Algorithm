\documentclass[cs4size,a4paper,adobefonts]{ctexart}
\usepackage{amsmath,amsthm,amssymb}
\usepackage[colorlinks=true,allcolors=black]{hyperref}
\usepackage{indentfirst}
\usepackage[a4paper,left=2.5cm,right=2.5cm,bottom=2.5cm,top=2.5cm]{geometry}
\usepackage{fontspec}
\setmainfont{Palatino}
\setmonofont[Scale=MatchLowercase]{Monaco}
\pagestyle{plain}
\punctstyle{kaiming}
\usepackage{unicode-math}
\setmathfont{Asana Math}

\newcommand{\GridMaze}{\href{http://itunes.apple.com/app/grid-maze/id553265800?mt=8}{Grid Maze}}

\begin{document}
\title{\bfseries 算法有什么用}
\author{\href{mailto:txyyss@gmail.com}{王盛颐}}
\date{}
\maketitle

\section*{缘起}
最近我的 iPad 程序 \GridMaze{} 在历经了 5 个月的开发,13 天的审核之后,
终于在苹果应用商店上架了。这个程序的主要功能是能根据输入的文字或图案,
生成一个迷宫,使得走出这个迷宫的唯一路径能形成当初输入的文字或图案。

生成这样一个迷宫的想法最早可以追溯到 2007 年底,不过那时完全不知道该怎
么做,于是这个想法也就是在脑子里徘徊了一阵,就置之一旁了。直到一年后也
就是 2008 年底,在看到一篇根据图形生成迷宫轮廓的文章
\cite{Xu:2007:ImageMaze}时,我突然想明白该怎么做了,于是就用
Mathematica 做了一些试验验证了我的想法,之后就一边完善想法一边写程序,
直到做出了一个以我名字为解法的迷宫。这个原型程序也就因为目标已达成而被
束之高阁。2010年,在朋友的鼓励下,我用 Qt 写了这个迷宫生成程序的界面,
同时用 C++ 重写了原先用 Java 和 Mathematica 写的部分,完善了一些功能,
这就是
\href{https://sites.google.com/site/txyyss/projects/text-maze-creator}{Text
  Maze Creator}。这个程序依赖一个第三方的 TSP 求解程序和一个
ActionScript 的编译器,操作起来很复杂,所以也没有提供下载,只是放了一些
样例在网上。转眼就到了2012 年,由于陆续能收到请求生成迷宫的邮件,我决定
写一个大家都能用的迷宫生成程序。我重新设计了界面,自己写了 TSP 的求解算
法,这才有了 iPad 上的 \GridMaze。

在开发 \GridMaze{} 的过程中,我遇到了很多有意思的问题,为了解决这些问题
参考和设计了很多算法。我觉得有必要把这些问题和解决办法整理一番,也算是
在初步完成这个项目之后,做一个总结报告。

\bibliographystyle{plain}
\bibliography{algorithm}
\end{document}
