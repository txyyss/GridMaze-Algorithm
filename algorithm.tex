\documentclass[cs4size,a4paper,adobefonts]{ctexart}
\usepackage{amsmath,amsthm,amssymb}
\usepackage[colorlinks=true,linkcolor=black]{hyperref}
\usepackage{indentfirst}
\usepackage[a4paper,left=2.5cm,right=2.5cm,bottom=2.5cm,top=2.5cm]{geometry}
\usepackage{fontspec}
\setmainfont{Palatino}
\setmonofont[Scale=MatchLowercase]{Monaco}
\pagestyle{plain}
\punctstyle{kaiming}
\usepackage{unicode-math}
\setmathfont{Asana Math}
\begin{document}
\title{\bfseries 算法有什么用}
\author{\href{mailto:txyyss@gmail.com}{王盛颐}}
\date{}
\maketitle

\section*{缘起}
最近我的 iPad 程序
\href{http://itunes.apple.com/app/grid-maze/id553265800?mt=8}{Grid
  Maze} 在历经了 5 个月的开发,13 天的审核之后,终于在苹果应用商店上架
了。这个程序的主要功能是能根据输入的文字或图案,生成一个迷宫,使得走出
这个迷宫的唯一路径能形成当初输入的文字或图案。

生成这样一个迷宫的想法最早可以追溯到 2007 年底,不过那时完全不知道该怎
么做才好,于是也就是在脑子里徘徊了会儿,就放在一边了。2008 年的下半年,
在看到一篇根据图形生成迷宫轮廓的文章\cite{Xu:2007:ImageMaze}时,我突然
想明白该怎么做了,于是就拿 Mathematica 做了一些实验验证了我的想法。

\bibliographystyle{plain}
\bibliography{algorithm}
\end{document}
